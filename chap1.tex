\chapter{\label{intro}Introduction}

\setcounter{equation}{0}
\setcounter{table}{0}
\setcounter{figure}{0}
\setcounter{chapter}{1}

\baselineskip 24pt
\hspace{10pt}
\\
Based on the work for a detector at CLIC, this report provides a conceptual description and illustration of the CLD detector. CLD is one of the detectors planned for a future 100-kilometer $e+e-$ linear circular collider (FCC-ee). The note also includes  a brief description of the simulation and
reconstruction tools used in the linear collider community, which have been adapted for
physics and performance studies of CLD. We have learned about detector description which is an 
essential component  to analyze data resulting from
particle collisions in high energy physics experiments.
 
In this project, We have selected the \emph{Higgs-Strahlung process}$(e^{+}+ e^{-}\longrightarrow Z+ 
H)$  by  $e^{+}e^{-}$ collision  at $240GeV$ center of mass energy. We have 
run a fast parametric detector simulation with Delphes in the EDM4Hep format
then applied  event selection on those samples with FCCAnalyses
and produced flat ntuples with observables of interest with FCCAnalyses.
Then we have analyzed the different kinematics of $Higgs$ and $Z$ boson and their daughter particles
with FCCAnalyses.

As we know $Z$ boson is the exchange particles mediating weak interaction, which can decay when it 
produces. We have only focused on two of the  detectable decay-modes, the decay into 
electron-positron$(e^{+}e^{-})$ or muon-antimuon$(\mu^{+}\mu^{-})$. 
We have studied different kinematic parameter such as energy$(E)$, transverse momentum$(p_T)$, 
momentum $(p)$, rapidity$(y)$, $\theta$ of daughter particle and  reconstructed its 
four momentum $(p^{\mu})$. In order to satisfies invariant mass law we have reconstructed the mass  of  $Z$ boson from their daughter particles.



% The toolkit will be built
% reusing already existing components from the ROOT geometry package and provides missing
% functional elements and interfaces to offer a complete and coherent detector description solution.
% A natural integration to Geant4, the detector simulation program used in high energy physics,
% is provided.

